\section{Parameter Estimation Using EM}

We need to find optimal $\theta$ that maximizes $log (p(y,q|\theta))$ where
$\theta = (\pi,A,\eta)$. We know $\pi$ and $A$. $\eta$ represents parameters
necessary for $p(y_t|q_t)$, the output distribution.

Let's write the full $p(y,q)$ distribution;

\begin{eqnarray*}
p(q,y) &=& \pi_{q_0} \prod_{t=0}^{T-1} a_{q_t,q_{t+1}} \prod_{t=0}' p(y_t|q_t,\eta)
\end{eqnarray*}


\begin{proof}
\begin{eqnarray*}
\prod_{t=0}^{T-1} a_{q_t,q_{t+1}} &=& p(q_0)p(q_1|q_0)p(q_2|q_1)...\\
&=& p(q_1,q_0)p(q_2|q_1)..\\
&=& p(...,q_2,q_1,q_0) \\
&=& p(q)
\end{eqnarray*}

Due to conditional independence,

\begin{eqnarray*}
p(y_0,y_1,y_2,...|q_0,q_1,q_2,..) &=& p(y_0|q_0)p(y_1|q_1)p(y_2|q_2)...\\
&=& \prod_{t=0}' p(y_t|q_t)
\end{eqnarray*}

Finally

\begin{eqnarray*}
p(y_0,y_1,y_2,...|q_0,q_1,q_2,..)p(q)  &=& p(y,q)
\end{eqnarray*}
\end{proof}

We need to bring $a_{ij}$ in to the joint distribution.

Trick: Let's write $a_{q_t,q_{t+1}}$ as

\begin{eqnarray*}
a_{q_t,q_{t+1}} &=& \prod_{i,j=1}^M[a_{ij}]^{q_t^i q_{t+1}^j}
\end{eqnarray*}
Why does this work? Remember: $[a_{ij}]$ is a scalar, $[a_{ij}]^1 = a_{ij}$ and
$[a_{ij}]^0 = 1$. $q_t^i q_{t+1}^j$ can only be $1$ if they are both $1$ at the
same time, that is, $q_t^i$ and $q_{t+1}^j$ are selecting the correct state.

We use the same technique for $p(y_t|q_t)$ which is represented by matrix
$\eta$, then

\begin{eqnarray*}
p(y_t|q_t) &=& \prod_{i,j=1}^M [\eta_{ij}]^{q_t^i y_{t+1}^j}
\end{eqnarray*}

Then log likelihood $\log p(q,y)$ is;

{\small
\begin{eqnarray*}
 &=& \log
 \bigg\{
 \prod_{i=1}^{T-1}[\pi]^{q_0^i}
 \prod_{t=1}^{T-1} \prod_{i,j=1}^M[a_{ij}]^{q_t^i q_{t+1}^j}
 \prod_{t=1}^{T-1} \prod_{i,j=1}^M[\eta_{ij}]^{q_t^i y_{t+1}^j}
 \bigg\} \\
 &=& 
 \sum_{i=1}^{T-1}q_0^i\log \pi +
 \sum_{t=0}^{T-1} \sum_{i,j=1}^M q_t^i q_{t+1}^j \log a_{ij} +
 \sum_{t=0}^{T-1} \sum_{i,j=1}^M q_t^i y_{t+1}^j \log \eta_{ij}
\end{eqnarray*}}

For estimating $\hat{a}_{ij}$ we need to take derivative using Lagrange
multiplier, set to zero and solve.

{\small
\begin{eqnarray*}
\frac{\partial}{\partial a_{ij}} \log p(y,q)
&=&
\frac{\partial}{\partial a_{ij}}
\bigg(
\sum_{t=0}^{T-1} \sum_{i,j=1}^M q_t^i q_{t+1}^j \log a_{ij} -
\lambda (M-\sum_{i,j=1}^M a_{ij})
\bigg)
\end{eqnarray*}
}
As we see, other terms which did not contain $a_{ij}$ simply dropped out.

{\small
\begin{eqnarray*}
\frac{\partial}{\partial a_{ij}} \log p(y,q)
&=& \sum_{t=0}^{T-1} \sum_{i,j=1}^M \frac{q_t^i q_{t+1}^j}{a_{ij}} -
\lambda \sum_{i,j=1}^M 1 \\
&=&  \sum_{i,j=1}^M ( \sum_{t=0}^{T-1} \frac{q_t^i q_{t+1}^j}{a_{ij}} -
\lambda) = 0 \\
&=&  \sum_{t=0}^{T-1} \frac{q_t^i q_{t+1}^j}{a_{ij}} - \lambda = 0 \\
\lambda &=&  \sum_{t=0}^{T-1} \frac{q_t^i q_{t+1}^j}{a_{ij}} \\
\lambda a_{ij} &=&  \sum_{t=0}^{T-1} q_t^i q_{t+1}^j\\
\lambda \sum_{j=1}^M a_{ij} &=&  \sum_{j=1}^M \sum_{t=0}^{T-1} q_t^i q_{t+1}^j\\
\lambda &=&  \sum_{j=1}^M \sum_{t=0}^{T-1} q_t^i q_{t+1}^j\\
a_{ij} &=& \frac{\sum_{t=0}^{T-1} q_t^i q_{t+1}^j}{\lambda} \\
\hat{a}_{ij}  &=& \frac{\sum_{t=0}^{T-1} q_t^i q_{t+1}^j}{\sum_{j=1}^M \sum_{t=0}^{T-1} q_t^i q_{t+1}^j} 
\end{eqnarray*}}

Similarly for $\eta_{ij}$ and $\pi_i$,

\begin{eqnarray*}
\hat{\eta}_{ij} &=& \frac{\sum_{t=0}^{T-1} q_t^i y_t^j}{\sum_{k=1}^N
\sum_{t=0}^{T-1} q_t^i y_t^k} \\
\hat{\pi}_i &=& q_0^i
\end{eqnarray*}

If we define $m_{ij} \equiv \sum_{t=0}^{T-1} q_t^i q_{t+1}^j$ and $n_{ij}
\equiv \sum_{t=0}^{T-1} q_t^i y_t^j$ then,

\begin{eqnarray*}
\hat{a}_{ij}  &=& \frac{m_{ij}}{\sum_{j=1}^M m_{ij}}  \\
\hat{\eta}_{ij} &=& \frac{n_{ij}}{\sum_{k=1}^Nn_{ik}}\\
\hat{\pi}_i &=& q_0^i
\end{eqnarray*}

For the E-step we use expectations instead of the random variables themselves:

\begin{eqnarray*}
E(m_{ij}|y,\theta^{(p)}) = \sum_{t=0}^{T-1}E(q_t^i q_{t+1}^j|y,\theta^{(p)})
\end{eqnarray*}

Since random variable $q_t^i q_{t+1}^j$ is either 1 or 0, the expectation of
this random variable is equal to its probability

\begin{eqnarray*}
E(m_{ij}|y,\theta^{(p)}) &=& \sum_{t=0}^{T-1}p(q_t^i q_{t+1}^j|y,\theta^{(p)})\\
&=& \sum_{t=0}^{T-1} \xi_{t,t+1}^{ij}
\end{eqnarray*}
We plug this back into formula for $\hat{a}_{ij}$

\begin{eqnarray*}
\hat{a}_{ij}^{(p+1)}  &=& \frac{\sum_{t=0}^{T-1} \xi_{t,t+1}^{ij}}
    {\sum_{k=1}^M \sum_{t=0}^{T-1} \xi_{t,t+1}^{ik}}  \\
\end{eqnarray*}

Since $\xi(q_t,q_{t+1})=\xi_{t,t+1}=p(q_t,q_{t+1}|y)$ and $\sum_{k=1}^M \sum_{t=0}^{T-1}
\xi_{t,t+1}^{ik} = \gamma_t^i$,

\begin{eqnarray*}
\hat{a}_{ij}^{(p+1)}  &=& \frac{\sum_{t=0}^{T-1} \xi_{t,t+1}^{ij}}{\sum_{t=0}^{T-1} \gamma_t^i} 
\end{eqnarray*}

Using similar procedure we can also derive the following;

\begin{eqnarray*}
\hat{\pi}^{(p+1)} &=& \gamma_0^i\\
\hat{\eta}_{ij}^{(p+1)} &=& \frac{\sum_{t=0}^T\gamma_{t}^i y_t^j}{\sum_{t=0}^T\gamma_{t}^i}
\end{eqnarray*}

$\hat{\eta}_{ij}$ assumes a multinomial output distribution $y_t$. For stock
prediction purposes, we will be using a Gaussian distribution to represent
each stock price at time $t$. Therefore, we can simply use $y_t$ itself as the
output instead of the $y_t^j$, hence $\hat{\eta}$ changes to being the mean of
the estimated Gaussian.

\begin{eqnarray*}
\hat{C}_{i}^{(p+1)} &=& \frac{\sum_{t=0}^T\gamma_{t}^i y_t}{\sum_{t=0}^T\gamma_{t}^i}
\end{eqnarray*}

In each iteration

\begin{itemize}
   \item We calculate $\hat{a}$, $\hat{C}$, $\hat{R}$, $\hat{\pi}$ using given
     $q$ and $y$.
   \item We calculate $p(q_t|y)$ for each $q_t$ using $A$, $\eta$ and $\pi$ using
   forward backward recursion.
\end{itemize}
